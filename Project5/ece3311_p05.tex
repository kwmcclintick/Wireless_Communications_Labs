\documentclass[letterpaper,12pt]{article}

%-----------------------------------------------------------
% Packages
%-----------------------------------------------------------
\usepackage{graphicx}
\usepackage{amssymb}
\usepackage{amsmath}
\usepackage{amstext}
\usepackage{array}
\usepackage{subfigure}
\usepackage{color}
\usepackage{listings}
\usepackage[framed]{mcode}
\usepackage{caption}
%-----------------------------------------------------------


%-----------------------------------------------------------
% Report dimensions
%-----------------------------------------------------------
\addtolength{\oddsidemargin}{-.875in}
\addtolength{\evensidemargin}{-.875in}
\addtolength{\textwidth}{1.75in} \addtolength{\topmargin}{-.875in}
\addtolength{\textheight}{1.75in} \special
{papersize=\the\paperwidth,\the\paperheight}
%-----------------------------------------------------------

% Macro definitions
\newcommand{\BDE}{{\sc bde}}
\newcommand{\FDS}{{\sc fds}}
\newcommand{\SPW}{{\sc spw}}
\newcommand{\goodgap}{%
\hspace{\subfigtopskip}%
\hspace{\subfigbottomskip}}
\newlength{\tmptmp}
\newcommand{\cconv}[1]{%
\mathchardef\lll="320D%
\settowidth{\tmptmp}{$#1$}\setlength{\unitlength}{\tmptmp}\;\raisebox{0\height}{\begin{picture}(1,1)(0,0)%
   \put(0,0){\scalebox{1.3}{$\lll$}}\put(.18,.0){\scalebox{.65}{$#1$}}%
  \end{picture}} }

% Define counters
\newcounter{questioncnt}
\newcounter{matlabcnt}
\setcounter{questioncnt}{1}
\setcounter{matlabcnt}{1}


\begin{document}

\begin{table}[h]
\begin{tabular}{m{2.5in}m{0.15in}m{3.95in}}
  \includegraphics[width=2.5in,keepaspectratio=true]{images/logo.eps} &
  {} &
  \begin{tabular}{c}
    {\large ECE3311 -- Principles of Communication} \\
    \vspace*{2mm}{\large Systems}\\
    \vspace*{2mm}{\normalsize Project~05}\\
    {\normalsize Due: 5:00 PM, Wednesday 13 December 2017}
  \end{tabular}\\
  \hline
\end{tabular}
\end{table}

%-----------------------------------------------------------
\section{Project Objective \& Learning Outcomes}

In this project, we will investigate the physical layer of a
generic digital communication system using
Orthogonal Frequency Division Multiplexing (OFDM) over channels
that introduce noise.
From this project, it is expected that the following learning outcomes are achieved:
\begin{itemize}
 \item Understand how multicarrier modulation operates by sending multiple signals across different carrier frequencies at the same time.
 \item Learn about the orthogonal frequency division multiplexing concept by leveraging the Inverse Discrete Fourier Transform (IDFT) and DFT.
 \item Master Quadrature Amplitude Modulation (QAM) and how it operates as a 2-dimensional modulation scheme.
\end{itemize}

\section{Modulation and Demodulation of QAM symbols}

Quadrature Amplitude Modulation (QAM) is the technique of
transmitting data on two quadrature carriers (\textit{i.e.}, they are
$90^{\circ}$ out of phase with each other, making them {\it
orthogonal}).
In what follows, we will consider a digital implementation of the modulation,
so that the carrier signals are
 $\cos(\omega_kn)$ and
$\sin(\omega_kn)$, with carrier frequency $\omega_k$. The
amplitude of each of these two carriers is specified by the
sequence of input bits, and is changed every $2N$ samples, which
is called the {\em QAM symbol period}. The amplitudes can take on
only a finite number of possible values.
%The digital information is thus contained in the amplitudes of the two carrier signals. Of course, in order to enable decoding, the time over which these amplitudes  remain constant (\textit{i.e.}, the {\em symbol period}, denoted here by $2N$)  has to be at least as big as  the carrier period.
%As seen in
Figure~\ref{qam_mod} represents a discrete-time model of the QAM
modulator, where
the sequence of input bits $d[m]$ is used to determine the amplitudes of the
two carriers. %The amplitudes can take on only a finite number of values,
              %typically some integer poerw of 2, say $2^D$. So f%
For each symbol period, $D$ bits of input are taken from the input
bit stream $d[m]$ and used to select one of the $2^D$ combinations
of amplitudes for the two carriers. Calling these amplitudes
$a[\ell]$ and $b[\ell]$ respectively, where $\ell$ represents the
symbol time index, these amplitudes are kept constant for the
duration of a symbol period by the upsampling and rectangular
window filtering shown in Figure~\ref{qam_mod}, yielding the
piecewise constant signals $a'[n]$ and $b'[n]$.
One can then write the expression of the modulated signal as   % the input bit sequence $d[m]$ is
%demultiplexed using a commutator into two data streams and encoded
%into an amplitude depending on the combination of bits. These
%amplitudes, $a_l$ and $b_l$, which remain constant over the symbol
%period of $2N$, are then modulated with the carriers, yielding
\begin{equation}\label{basic_qam_defn}
%\begin{array}{ccc}
  s[n]=a'[n]\cos(\omega_kn)+b'[n]\sin(\omega_kn)% & \hspace*{3mm} & \begin{array}{c}
%    n=0,\dots,2N-1 \\
%    0\le{k}<N \\
%  \end{array} \\
%\end{array}
\end{equation}
where $\omega_k=2\pi{k}/2N$ is the carrier frequency, and $2N$ is
the period of the symbol. Note that we have to limit the possible carrier frequencies to integer multiples of $2\pi/2N$ in order to keep orthogonality between the sinusoidal and cosinusoidal carriers in a digital implementation (see derivations below).

\begin{figure}[h]
\centering
%\begin{tabular}{cc}\subfigure{\includegraphics[width=.5\textwidth]{images/qam_mod.eps}}&\subfigure{
\scalebox{.5}{\input{images/qam_mod.pstex_t}}%}
%\end{tabular}\\
\caption{Rectangular QAM modulator  (multirate model of a digital implementation).\label{qam_mod}}
\end{figure}

\begin{description}
    \item[{\bf Question \arabic{questioncnt}\stepcounter{questioncnt} (5 points):}]
    Prove that the carriers $\cos(\omega_kn)$ and $\sin(\omega_kn)$ are %equal to zero when multiplied together and summed up across
orthogonal over the symbol period\footnote{Two real sequences $h[n]$ and $g[n]$ are said to be orthogonal over a period $0$ to $P$ if $\sum_{n=0}^{P-1} h[n]g[n]=0$.} . Are there any values of $k$ that are exceptions? Why?
\end{description}

\begin{description}
    \item[{\bf Matlab Task \arabic{matlabcnt}\stepcounter{matlabcnt} (10 points):}]
    Implement the QAM modulator as shown in Figure~\ref{qam_mod}, given an arbitrary $\omega_k$.
\end{description}



%As for the encoding of bits into amplitude values, if the QAM symbol
%uses $D$ bits, the in-phase and quadrature components each use $D/2$
%bits. Thus, for $2^D$ possible QAM symbols given a grouping of $D$
%bits, there are $2^{D/2}$ possible amplitude values for the in-phase
%components and $2^{D/2}$ possible amplitude values for the quadrature
%%components.
%How these amplitudes are chosen depends on several
%requirements, such as receiver complexity and error robustness.
In this project, we will be dealing with rectangular QAM signal
constellations, such as those shown in Figures~\ref{4qam_const},
\ref{16qam_const}, and \ref{64qam_const} for 4-QAM, 16-QAM, and
64-QAM, respectively. This bascially amounts to saying that
$a[\ell], b[\ell] \in \{\pm (2k-1)E, k=1,\ldots,2^{D/2-1}\}$,
where $E$ is some positive constant that scales the energy of the
sent signals, and $D/2$ is the number of bits used to represent
the amplitude level of one of the carriers during a symbol.

\begin{figure}
\begin{center}
\subfigure[4-QAM signal
constellation]{\label{4qam_const}\includegraphics{images/f1a.eps}}\goodgap
\subfigure[Rectangular 16-QAM signal
constellation]{\label{16qam_const}\includegraphics{images/f1b.eps}}\\
\subfigure[Rectangular 64-QAM signal
constellation]{\label{64qam_const}\includegraphics{images/f1c.eps}}
\caption{Three types of QAM signal constellations. The in-phase
values indicate the amplitude of the carrier $\cos(\omega_kn)$
while the quadrature values indicate the amplitude of the carrier
$\sin(\omega_kn)$.}
\end{center}
\end{figure}

One of the advantages of QAM signalling is the fact that
demodulation is relatively simple to perform. From
Figure~\ref{qam_demod}, the received signal, $r[n]$, is split into
two streams and each multiplied by carriers, $\cos(\omega_kn)$ and
$\sin(\omega_kn)$, followed by a summation block (implemented here through
filtering by a
rectangular window followed by downsampling). This process
produces estimates of the in-phase and quadrature amplitudes,
$\hat{a}[l]$ and $\hat{b}[l]$, namely
\begin{equation}\label{a_k_est}\begin{split}
\hat{a}[\ell]&=\sum\limits_{n=2\ell{N}}^{2\ell{N}+2N-1}r[n]\cos(\omega_kn)\\
&=\sum\limits_{n=2\ell{N}}^{2\ell{N}+2N-1}a'[n]\cos(\omega_kn)\cos(\omega_kn)+\sum\limits_{n=2\ell{N}}^{2\ell{N}+2N-1}b'[n]\sin(\omega_kn)\cos(\omega_kn)\\
&=\sum\limits_{n=2\ell{N}}^{2\ell{N}+2N-1}a'[n]\cos\left(\frac{2\pi{k}{n}}{2N}\right)\cos\left(\frac{2\pi{k}{n}}{2N}\right)+\sum\limits_{n=2\ell{N}}^{2\ell{N}+2N-1}b'[n]\sin\left(\frac{2\pi{k}{n}}{2N}\right)\cos\left(\frac{2\pi{k}{n}}{2N}\right)\\
&=\sum\limits_{n=2\ell{N}}^{2\ell{N}+2N-1}\frac{a'[n]}{2}
\end{split}\end{equation} and
\begin{equation}\label{b_k_est}\begin{split}
\hat{b}[\ell]&=\sum\limits_{n=2\ell{N}}^{2\ell{N}+2N-1}r[n]\sin(\omega_kn)\\
&=\sum\limits_{n=2\ell{N}}^{2\ell{N}+2N-1}a'[n]\cos(\omega_kn)\sin(\omega_kn)+\sum\limits_{n=2\ell{N}}^{2\ell{N}+2N-1}b'[n]\sin(\omega_kn)\sin(\omega_kn)\\
&=\sum\limits_{n=2\ell{N}}^{2\ell{N}+2N-1}a'[n]\cos\left(\frac{2\pi{k}{n}}{2N}\right)\sin\left(\frac{2\pi{k}{n}}{2N}\right)+\sum\limits_{n=2\ell{N}}^{2\ell{N}+2N-1}b'[n]\sin\left(\frac{2\pi{k}{n}}{2N}\right)\sin\left(\frac{2\pi{k}{n}}{2N}\right)\\
&=\sum\limits_{n=2\ell{N}}^{2\ell{N}+2N-1}\frac{b'[n]}{2}
\end{split}\end{equation}
where, due to the orthogonality of the two carriers, the cross
terms vanish, leaving the desired amplitude (after some
trigonometric manipulation). %Then, the estimates $\hat{a}_l$ and
%$\hat{b}_l$ are passed into a decision maker, which uses a
%nearest-neighbour rule to find the most likely known amplitude
%level and decode it into a group of bits.
 The bits etsimated from $\hat{a}[\ell]$ and $\hat{b}[\ell]$  bits are then
multiplexed together, forming the reconstructed
version of $d[m]$, $\hat{d}[m]$.

\begin{figure}[t]
\centering
%\begin{tabular}{cc}\subfigure{\includegraphics[width=.5\textwidth]{images/qam_demod.eps}}&\subfigure{
\scalebox{.5}{\input{images/qam-demod.pstex_t}}%}
%\end{tabular}
\caption{Rectangular QAM demodulator.}\label{qam_demod}
\end{figure}

\begin{description}
    \item[{\bf Question \arabic{questioncnt}\stepcounter{questioncnt} (5 points):}]
    Prove that Eqs.~(\ref{a_k_est}) and (\ref{b_k_est}) are true. What are the exceptions
    for the carrier frequencies for perfect recovery% under ideal conditions
? Why is $k$  constrained to be
    below $N$?
\end{description}

\begin{description}
    \item[{\bf Matlab Task \arabic{matlabcnt}\stepcounter{matlabcnt} (15 points):}]
    Implement the rectangular QAM demodulator %and decision maker,
    as shown in Figure~\ref{qam_demod}, given an arbitrary
    $\omega_k$. Test the cascade of the QAM modulator and
    demodulator to see if the input is completely recovered at the
    output. Implement the constellations of $4$-QAM, $16$-QAM and
    $64$-QAM. also implement the special, degenerate case of BPSK, where only
    the cosine carrier is modulated with one bit.
\end{description}


We have so far dealt with modulation and demodulation in an ideal
setting. Thus, one should expect that $\hat{d}[m]=d[m]$ with
probability of 1. However, in the next subsection we will examine
a physical phenomenon that distorts the transmitted signal,
resulting in transmission errors.


\section{Noise}\label{noisesubsection}

By definition, {\it noise} is an undesirable disturbance
accompanying the received signal that may distort the information
carried by the signal. Noise can originate from human-made and
natural sources, such as thermal noise due to the thermal
agitation of electrons in transmission lines, antennas, or other
conductors.

The combination of such sources of noise is known to have a
Gaussian distribution, as shown in Figure~\ref{awgn_hist}. A
histogram of zero-mean Gaussian noise with a variance of
$\sigma_n^2=0.25$ is shown, with the corresponding continuous
probability density function (pdf) superimposed on it. The
continuous Gaussian pdf is defined as
\begin{equation}\label{gaussian_pdf}
f_X(x)=\frac{1}{\sqrt{2\pi\sigma_n^2}}e^{-\frac{(x-\mu_n)^2}{2\sigma_n^2}}
\end{equation}
where $\mu_n$ and $\sigma_n^2$ are the mean and variance.

In this work, we will also make the assumption that the noise
introduced by the channel is {\it white}, which means that the
noise has frequency content that is approximately flat, as shown
in Figure~\ref{awgn_psd} for the case of zero-mean white Gaussian
Noise with a variance of $\sigma_n^2=0.25$. Although in reality
this assumption may not hold in certain cases, it does help
simplify the process of demodulation.
\begin{figure}[t]
\centering\begin{tabular}{cc}
\subfigure[Histogram of AWGN superimposed on the probability
density function for a Gaussian random variable of
N(0,0.25)]{\label{awgn_hist}\includegraphics{images/f5.eps}}&%\goodgap
\subfigure[Power spectral density of AWGN with
N(0,0.25)]{\label{awgn_psd}\includegraphics{images/f6.eps}}
\end{tabular}
\caption{Time and frequency domain properties of AWGN}
\end{figure}

Thus, when noise is added to the transmitted signal, the receiver
has to make a decision on what has been transmitted based on the
received signal. Usually this is accomplished via a ``nearest
neighbour'' rule with a known set of symbols. However, if a large
amount of noise is added to the signal, there is a possibility
that the received symbol might be shifted closer to a symbol other
than the correct one, resulting in an error. As seen in
Figures~\ref{4qam_little noise}, \ref{4qam_moderate_noise}, and
\ref{4qam_heavy_noise}, where additive white Gaussian noise is
included with the signal, it is readily observable that as the
noise power increases, the constellation points become fuzzy until
they begin overlapping with each other. It is at this point that
the system will begin to experience errors.
\begin{figure}[t]
\begin{center}
\subfigure[AWGN $N(0,0.00001)$]{\label{4qam_little noise}\includegraphics[width=.3\textwidth]{images/f4a.eps}}\goodgap \subfigure[AWGN $N(0,0.01)$]{\label{4qam_moderate_noise}\includegraphics[width=.3\textwidth]{images/f4b.eps}}
\subfigure[AWGN
$N(0,0.25)$]{\label{4qam_heavy_noise}\includegraphics[width=.3\textwidth]{images/f4c.eps}}
\caption{16-QAM signal constellations with varying amounts of
noise}
\end{center}
\end{figure}

\begin{description}
    \item[{\bf Matlab Task \arabic{matlabcnt}\stepcounter{matlabcnt} (10 points):}]
    Implement an AWGN generator that accepts as inputs: the
    variance $\sigma^2$, the mean $\mu$, and the number of random
    points $R$. Verify that it works by creating a histogram of
    its output and compare against the Gaussian pdf as described
    by Eq.~(\ref{gaussian_pdf}) for $\mu=3$, $\sigma^2=2$.
\end{description}



\section{Probability of Bit Error}\label{ber_section}

The introduction of noise, which is a stochastic signal, to the
transmitted signal means that the overall received signal is also
stochastic. Thus, transmitting the same information over and over
again may yield different results each time, assuming  that the
same modulation scheme and receiver design are employed.

In order to quantify the performance of a particular system set-up
(\textit{e.g.}, modulation scheme, receiver design, etc...), many use the
ratio between the number of bit errors that occur and the total
number of bits transmitted. This ratio is an estimate of the {\it
Probability of Bit Error} or {\it Bit Error Rate} (BER) and is
dependent on the noise variance, the modulation scheme employed,
and the receiver design.
\begin{figure}[t]
  \centering
  \includegraphics{images/f7.eps}\\
  \caption{Probability of Bit Error curves for 4-QAM, rectangular 16-QAM, and rectangular 64-QAM}\label{ber_curves}
\end{figure}

In Figure~\ref{ber_curves}, the BER curves for rectangular 4-QAM,
16-QAM, and 64-QAM are presented when additive white Gaussian
noise (AWGN) is introduced. Notice how as the signal-to-noise
ratio (SNR) increases (noise power decreases), the probability of
bit error decreases, as expected. Note that the SNR is the ratio
of symbol energy to the noise energy, namely
\begin{equation}
\gamma=\frac{E_s}{\sigma_n^2},%. \mbox{\color{red}DEFINE MORE RIGOROUSLY}
\end{equation}where $E_s=\frac{1}{N}\sum_{n=0}^{2N-1} s^2[n]$.

\begin{description}
    \item[{\bf Matlab Task \arabic{matlabcnt}\stepcounter{matlabcnt} (10 points):}]
    Verify that rectangular QAM modulator/demodulator works by
    introducing the appropriate amount of noise and comparing the
    resulting BER with the BER in Figure~\ref{ber_curves} at
    $10^{-3}$. What values of SNR did you employ?
\end{description}



\section{The OFDM Principle}
\begin{figure} \centering
%\includegraphics[height=.7\textheight]{images/oqam_sys.eps}\\
\scalebox{.5}{\input{images/oqam_sys.pstex_t}}
\caption{Transmitter of an orthogonally multiplexed QAM
system}\label{oqamsys}
\end{figure}

As we have seen thus far, signals can be transmitted on a single
carrier frequency. For instance, Eq.~(\ref{basic_qam_defn})
modulates to the carrier frequency $\omega_k$. However, the
transmitted signal may only use up a small portion of the total
available bandwidth. To increase bandwidth efficiency and
throughput, it is possible to send additional QAM signals in other portions
of the unused bandwidth simultaneously. An example of this is
shown in Figure~\ref{oqamsys}, where we have taken the QAM
modulator of Figure~\ref{qam_mod} and put several of them in
parallel, each with a different carrier frequency. The data for
each modulator came from portions of a high speed bit stream. This
is principle behind multicarrier modulation.

\begin{description}
    \item[{\bf Question \arabic{questioncnt}\stepcounter{questioncnt} (5 points):}]
    Draw a schematic for an %orthogonally
    multiplexed QAM
    demodulator as in Figure~\ref{oqamsys}. Show that the subcarriers of the orthogonally multiplexed
    QAM system are orthogonal.
\end{description}


\begin{description}
    \item[{\bf Matlab Task \arabic{matlabcnt}\stepcounter{matlabcnt} (15 points):}]
    Implement Figure~\ref{oqamsys} and its corresponding receiver.
    Verify that it works under ideal conditions. What carrier
    frequencies did you not use in the implementation?
\end{description}


Orthogonal Frequency Division Multiplexing (OFDM) is an efficient
type of multicarrier modulation, which employs the discrete
Fourier transform (DFT) and inverse DFT (IDFT) to modulate and
demodulate the data streams.
Since the carriers used in Figure~\ref{oqamsys} are sinusoidal function of
$2\pi k n/2N$, it should come as no surprise that a $2N$-point DFT or IDFT can
carry out the same modulation, since it contains also summations of terms of
the form $e^{\pm2\pi kn/2N}$.
The set-up of an OFDM system is
presented in Figure~\ref{ofdmsys}. A high-speed digital input,
$d[m]$, is demultiplexed into $N$ subcarriers using a commutator.
The data on each subcarrier is then modulated into an $M$-QAM
symbol, which maps a group of $\log_2(M)$ bits at a time. Unlike
the %baseband
 representation of Eq.~(\ref{basic_qam_defn}), for
subcarrier $k$ we will rearrange $a_k[\ell]$ and $b_k[\ell]$ into real and
imaginary components such that the output of the ``modulator'' block  is
$p_k[\ell]=a_k[\ell]+jb_k[\ell]$. In order for the output of the IDFT block to be
real, given $N$ subcarriers we must use a $2N$-point IDFT, where
terminals $k=0$ and $k=N$ are ``don't care'' inputs. For the
subcarriers $1\le{k}\le{N-1}$, the inputs are $p_k[\ell]=a_k[\ell]+jb_k[\ell]$,
while for the subcarriers $N+1\le{k}\le{2N-1}$, the inputs are
$p_{k}[\ell]=a_{2N-k}[\ell]+jb_{2N-k}[\ell]$.

\begin{description}
    \item[{\bf Question \arabic{questioncnt}\stepcounter{questioncnt} [BONUS] (3 points):}]
    Why are $k=0$ and $k=N$ ``don't care'' inputs?
\end{description}

The IDFT is then performed, yielding
\begin{equation}
s[2\ell{N}+n]=\frac{1}{2N}\sum\limits_{k=0}^{2N-1}p_k[\ell]e^{j(2\pi{nk}/2N)},
\end{equation}where this time $2N$ consecutive samples of $s[n]$ constitute an
OFDM symbol, which is a sum of $N$ different QAM symbols.

This results in the data being modulated on several subchannels.
This is achieved by multiplying each data stream by a
$\sin(Nx)/\sin(x)$, several of which are shown in
Figure~\ref{freq_resp_ofdm}.

The subcarriers are then multiplexed together using a commutator,
forming the signal $s[n]$, and transmitted to the receiver. Once
at the receiver, the signal is demultiplexed into $2N$ subcarriers
of data, $\hat{s}[n]$, using a commutator and a $2N$-point DFT,
defined as
\begin{equation}
\bar{p}_k[\ell]=\sum\limits_{n=0}^{2N-1}\hat{s}[2\ell{N}+n]e^{-j(2\pi{nk}/2N)},
\end{equation}
is applied to the inputs, yielding the estimates of $p_k[\ell]$,
$\bar{p}_k[\ell]$. The output of the equalizer, $\hat{p}_k[\ell]$,
then passed through a demodulator and the result multiplexed
together using a commutator, yielding the reconstructed high-speed
bit stream, $\hat{d}[m]$.
\begin{figure}
\centering
\includegraphics{images/ofdm_system.eps}
\caption{Overall schematic of an Orthogonal Frequency Division
Multiplexing System}\label{ofdmsys}
\end{figure}


\begin{figure}
\centering
\includegraphics{images/f8.eps}
\caption{Characteristics of Orthogonal Frequency Division
Multiplexing: Frequency response of OFDM
subcarriers.}\label{freq_resp_ofdm}
\end{figure}

\begin{description}
    \item[{\bf Question \arabic{questioncnt}\stepcounter{questioncnt} [BONUS] (5 points):}]
    Show that the OFDM configuration and the orthogonally
    multiplexed QAM system are identical. Which of the two
    implementations is faster if the OFDM system employs FFTs?
    Prove it by comparing the number of operations.
\end{description}

\begin{description}
    \item[{\bf Matlab Task \arabic{matlabcnt}\stepcounter{matlabcnt} [BONUS] (15 points):}]
    Implement Figure~\ref{ofdmsys} using IFFT and FFT blocks using {\tt ifft} and {\tt fft}. The
    size of the input/output should be a variable.
\end{description}



%-----------------------------------------------------------
\section{Final Report Format \& Content}

Each experiment report should possess the following format:
\begin{itemize}
 \item A cover page (2 points) that includes the course number, project number, names and WPI ID numbers, submission date.
 \item A narrative (10 points) of the process taken during this experiment and the experiences encountered by the student.  Figures, plots, schematics, diagrams, snippets of source code, and other visuals are highly encouraged.
 \item Responses to all questions indicated in the project handout.   Please make sure that the responses are of sufficient detail.
 \item A summary (5 points) that contains all lessons learned from this project.
 \item All source code (5 points) generated (as an appendix).
\end{itemize}

This single document must be electronically submitted in {\bf PDF format} (no other formats will be accepted) via the ECE3311 CANVAS website by the due date. Failure to submit this report by the specified due date and time will result in a grade of ``0\%''.



\end{document}
