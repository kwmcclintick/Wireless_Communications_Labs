\documentclass[12pt]{article}

\usepackage{times,mathptm,amsmath,citesort,amssymb,graphicx,subfigure,indentfirst}

\addtolength{\oddsidemargin}{-.875in}
\addtolength{\evensidemargin}{-.875in}
\addtolength{\textwidth}{1.75in}

\addtolength{\topmargin}{-.875in}
\addtolength{\textheight}{1.75in}

\newcommand{\goodgap}{%
\hspace{\subfigtopskip}%
\hspace{\subfigbottomskip}}

% Define counters
\newcounter{questioncnt}
\newcounter{matlabcnt}
\setcounter{questioncnt}{1}
\setcounter{matlabcnt}{1}


\begin{document}

\title{High Speed Data Transmission over Telephone Network}
\author{ECSE 490A Term Project}
\date{Fall 2003}
\maketitle

\begin{abstract}
In this project, we will investigate the physical layer of a
generic Digital Subscriber Line (xDSL) modem. In particular, we
will study how the system transmits digital information using
Orthogonal Frequency Division Multiplexing (OFDM) over channels
that introduce both noise and fading due to multipath propagation.
With the details outlined in this handout, one should be able to
implement an OFDM-based generic xDSL system in Matlab and transmit
information across a non-ideal channel.
\end{abstract}


\section{Digital Communications Primer}

Most modern communication systems, including all Digital
Subscriber Line (xDSL) modems, use digital signalling techniques
to transmit large quantities of information over a short period of
time while being robust to transmission noise. The fundamental
unit of information in these digital communication systems is the
{\it bit}, which consists of only two values (``on'' and ``off'',
``1'' and ``0'', etc...). However, these systems usually map, or
{\it modulate}, groups of bits into symbols prior to transmission.
At the receiver, each symbol is compared to a known set of symbols
and is {\it demodulated} into the group of bits corresponding to
the closest matching symbol. In the next subsection, we will look
at the modulation scheme that will be employed in this project.


\subsection{Modulation and Demodulation of QAM symbols}

Quadrature Amplitude Modulation (QAM) is the technique of
transmitting data on two quadrature carriers (i.e. they are
$90^{\circ}$ out of phase with each other, making them {\it
orthogonal}). Since we are considering an all-digital
implementation, these carriers, $\cos(\omega_kn)$ and
$\sin(\omega_kn)$ with carrier frequency $\omega_k$, have their
amplitudes specified by the sequence of input bits. As seen in
Figure~\ref{qam_mod}, the input bit sequence $d[m]$ is
demultiplexed using a commutator into two data streams and encoded
into an amplitude depending on the combination of bits. These
amplitudes, $a_l$ and $b_l$, which remain constant over the symbol
period of $2N$, are then modulated with the carriers, yielding
\begin{equation}\label{basic_qam_defn}
\begin{array}{ccc}
  s[n]=a_l\cos(\omega_kn)-b_l\sin(\omega_kn) & \hspace*{3mm} & \begin{array}{c}
    n=0,\dots,2N-1 \\
    0\le{k}<N \\
  \end{array} \\
\end{array}
\end{equation}
where $\omega_k=2\pi{k}/2N$ is the carrier frequency, and $2N$ is
the period of the symbol.
\begin{figure}[t]
\centering
\includegraphics{images/qam_mod.eps}\\
\caption{Rectangular QAM modulator}\label{qam_mod}
\end{figure}

\begin{description}
    \item[{\bf Question \arabic{questioncnt}\stepcounter{questioncnt}:}]
    Prove that the carriers $\cos(\omega_kn)$ and $\sin(\omega_kn)$ are equal to zero when multiplied together and summed up across the symbol period. Are there any values of $k$ that are exceptions? Why?
\end{description}

\begin{description}
    \item[{\bf Matlab Task \arabic{matlabcnt}\stepcounter{matlabcnt}:}]
    Implement the QAM modulator as shown in Figure~\ref{qam_mod}, given an arbitrary $\omega_k$.
\end{description}



As for the encoding of bits into amplitude values, if the QAM
symbol uses $D$ bits, the in-phase and quadrature components each
use $D/2$ bits. Thus, for $2^D$ possible QAM symbols given a
grouping of $D$ bits, there are $2^{D/2}$ possible amplitude
values for the in-phase components and $2^{D/2}$ possible
amplitude values for the quadrature components. How these
amplitudes are chosen depends on several requirements, such as
receiver complexity and error robustness. In this project, we will
be dealing with rectangular QAM signal constellations, such as
those shown in Figures~\ref{4qam_const} and \ref{16qam_const} for
4-QAM and 16-QAM, respectively.
\begin{figure}[t]
\begin{center}
\subfigure[4-QAM signal
constellation]{\label{4qam_const}\includegraphics{images/f1a.eps}}\goodgap
\subfigure[Rectangular 16-QAM signal
constellation]{\label{16qam_const}\includegraphics{images/f1b.eps}}
\caption{Two types of QAM signal constellations. The in-phase
values indicate the amplitude of the carrier $\cos(\omega_kn)$
while the quadrature values indicate the amplitude of the carrier
$\sin(\omega_kn)$.}
\end{center}
\end{figure}

\begin{figure}[t]
\centering
\includegraphics{images/qam_demod.eps}\\
\caption{Rectangular QAM demodulator}\label{qam_demod}
\end{figure}

One of the advantages of QAM signalling is the fact that
demodulation is relatively simple to perform. From
Figure~\ref{qam_demod}, the received signal, $r[n]$, is split into
two streams and each multiplied by carriers, $\cos(\omega_kn)$ and
$\sin(\omega_kn)$, followed by a summation block. This process
produces estimates of the in-phase and quadrature amplitudes,
$\hat{a}_l$ and $\hat{b}_l$, namely
\begin{equation}\label{a_k_est}\begin{split}
\hat{a}_l&=\sum\limits_{n=0}^{2N-1}r[n]\cos(\omega_kn)\\
&=\sum\limits_{n=0}^{2N-1}a_l\cos(\omega_kn)\cos(\omega_kn)+\sum\limits_{n=0}^{2N-1}b_l\sin(\omega_kn)\cos(\omega_kn)\\
&=\sum\limits_{n=0}^{2N-1}a_l\cos\left(\frac{2\pi{k}{n}}{2N}\right)\cos\left(\frac{2\pi{k}{n}}{2N}\right)+\sum\limits_{n=0}^{2N-1}b_l\sin\left(\frac{2\pi{k}{n}}{2N}\right)\cos\left(\frac{2\pi{k}{n}}{2N}\right)\\
&=\sum\limits_{n=0}^{2N-1}\frac{a_l}{2}
\end{split}\end{equation} and
\begin{equation}\label{b_k_est}\begin{split}
\hat{b}_l&=\sum\limits_{n=0}^{2N-1}r[n]\sin(\omega_kn)\\
&=\sum\limits_{n=0}^{2N-1}a_l\cos(\omega_kn)\sin(\omega_kn)+\sum\limits_{n=0}^{2N-1}b_l\sin(\omega_kn)\sin(\omega_kn)\\
&=\sum\limits_{n=0}^{2N-1}a_l\cos\left(\frac{2\pi{k}{n}}{2N}\right)\sin\left(\frac{2\pi{k}{n}}{2N}\right)+\sum\limits_{n=0}^{2N-1}b_l\sin\left(\frac{2\pi{k}{n}}{2N}\right)\sin\left(\frac{2\pi{k}{n}}{2N}\right)\\
&=\sum\limits_{n=0}^{2N-1}\frac{b_l}{2}
\end{split}\end{equation}
where, due to the orthogonality of the two carriers, the cross
terms vanish, leaving the desired amplitude (after some
trigonometric manipulation). Then, the estimates $\hat{a}_l$ and
$\hat{b}_l$ are passed into a decision maker, which uses a
nearest-neighbour rule to find the most likely known amplitude
level and decode it into a group of bits. These bits are then
multiplexed together using a commutator, forming the reconstructed
version of $d[m]$, $\hat{d}[m]$.

\begin{description}
    \item[{\bf Question \arabic{questioncnt}\stepcounter{questioncnt}:}]
    Prove that Eqs.~(\ref{a_k_est}) and (\ref{b_k_est}) are true. What are the exceptions
    for the carrier frequencies for perfect recovery under ideal conditions? Why is $k$ is constrained to be
    below $N$?
\end{description}

\begin{description}
    \item[{\bf Matlab Task \arabic{matlabcnt}\stepcounter{matlabcnt}:}]
    Implement the rectangular QAM demodulator and decision maker,
    as shown in Figure~\ref{qam_demod}, given an arbitrary
    $\omega_k$. Test the cascade of the QAM modulator and
    demodulator to see if the input is completely recovered at the
    output.
\end{description}


We have so far dealt with modulation and demodulation in an ideal
setting. Thus, one should expect that $\hat{d}_n=d_n$ with
probability of 1. However, in the next subsection we will examine
a physical phenomenon that distorts the transmitted signal,
resulting in transmission errors.


\subsection{Noise}\label{noisesubsection}

By definition, {\it noise} is an undesirable disturbance
accompanying the received signal that may distort the information
carried by the signal. Noise can be originate from human-made and
natural sources, such as thermal noise due to the thermal
agitation of electrons in conductors, such as the transmission
lines or antennas.

The combination of such sources of noise is known to have a
Gaussian distribution, as shown in Figure~\ref{awgn_hist}. A
histogram of zero-mean Gaussian noise with a variance of
$\sigma_n^2=0.25$ is shown, with the corresponding continuous
probability density function (pdf) superimposed on it. The
continuous Gaussian pdf is defined as
\begin{equation}\label{gaussian_pdf}
f_X(x)=\frac{1}{\sqrt{2\pi\sigma_n^2}}e^{-\frac{(x-\mu_n)^2}{2\sigma_n^2}}
\end{equation}
where $\mu_n$ and $\sigma_n^2$ are the mean and variance.

In this work, we will also make the assumption that the noise
introduced by the channel is {\it white}, which means that the
noise has frequency content that is approximately flat, as shown
in Figure~\ref{awgn_psd} for the case of zero-mean white Gaussian
Noise with a variance of $\sigma_n^2=0.25$. Although in reality
this assumption may not hold in certain cases, it does help
simplify the process of demodulation.
\begin{figure}[t]
\begin{center}
\subfigure[Histogram of AWGN superimposed on the probability
density function for a Gaussian random variable of
N(0,0.25)]{\label{awgn_hist}\includegraphics{images/f5.eps}}\goodgap
\subfigure[Power spectral density of AWGN with
N(0,0.25)]{\label{awgn_psd}\includegraphics{images/f6.eps}}
\caption{Time and frequency domain properties of AWGN}
\end{center}
\end{figure}

Thus, when noise is added to the transmitted signal, the receiver
has to make a decision on what has been transmitted based on the
received signal. Usually this is accomplished via a ``nearest
neighbour'' rule with a known set of symbols. However, if a large
amount of noise is added to the signal, there is a possibility
that the received symbol might be shifted closer to a symbol other
than the correct one, resulting in an error. As seen,
Figures~\ref{4qam_little noise}, \ref{4qam_moderate_noise}, and
\ref{4qam_heavy_noise}, where additive white Gaussian noise is
included with the signal, it is readily observable that as the
noise power increases, the constellation points become fuzzy until
they begin overlapping with each other. It is at this point that
the system will begin to experience errors.
\begin{figure}[t]
\begin{center}
\subfigure[AWGN N(0,0.00001)]{\label{4qam_little noise}\includegraphics{images/f4a.eps}}\goodgap \subfigure[AWGN N(0,0.01)]{\label{4qam_moderate_noise}\includegraphics{images/f4b.eps}}\\
\subfigure[AWGN
N(0,0.25)]{\label{4qam_heavy_noise}\includegraphics{images/f4c.eps}}
\caption{16-QAM signal constellations with varying amounts of
noise}
\end{center}
\end{figure}

\begin{description}
    \item[{\bf Matlab Task \arabic{matlabcnt}\stepcounter{matlabcnt}:}]
    Implement an AWGN generator that accepts as inputs: the
    variance $\sigma^2$, the mean $\mu$, and the number of random
    points $R$. Verify that it works by creating a histogram of
    its output and compare against the Gaussian pdf as described
    by Eq.~(\ref{gaussian_pdf}).
\end{description}



\subsection{Probability of Bit Error}

The introduction of noise, which is a stochastic signal, to the
transmitted signal means that the overall received signal is also
stochastic. Thus, transmitting the same information over and over
again may yield different results each time, assuming the that the
same modulation scheme and receiver design are employed.

In order to quantify the performance of a particular system set-up
(e.g. modulation scheme, receiver design, etc...), many use the
ratio between the number of bit errors that occur and the total
number of bits transmitted. This ratio is an estimate of the {\it
Probability of Bit Error} or {\it Bit Error Rate} (BER) and is
dependent on the noise variance, the modulation scheme employed,
and the receiver design.
\begin{figure}[t]
  \centering
  \includegraphics{images/f7.eps}\\
  \caption{Probability of Bit Error curves for 4-QAM, rectangular 16-QAM, and rectangular 64-QAM}\label{ber_curves}
\end{figure}

In Figure~\ref{ber_curves}, the BER curves for rectangular 4-QAM,
16-QAM, and 64-QAM are presented when additive white Gaussian
noise (AWGN) is introduced. Notice how as the signal-to-noise
ratio (SNR) increases (noise power decreases), the probability of
bit error decreases, as expected. Note that the SNR is the ratio
of symbol energy to the noise energy, namely
\begin{equation}
\gamma=\frac{E_s}{\sigma_n^2}.
\end{equation}

\begin{description}
    \item[{\bf Matlab Task \arabic{matlabcnt}\stepcounter{matlabcnt}:}]
    Verify that rectangular QAM modulator/demodulator works by
    introducing the appropriate amount of noise and comparing the
    resulting BER with the BER in Figure~\ref{ber_curves} at
    $10^{-3}$. What values of SNR did you employ?
\end{description}



\section{The OFDM Principle}
\begin{figure}[ht] \centering
\includegraphics{images/oqam_sys.eps}\\
\caption{Transmitter of an orthogonally multiplexed QAM
system}\label{oqamsys}
\end{figure}

As we have seen thus far, signals can be transmitted on single
carrier frequency. For instance, Eq.~(\ref{basic_qam_defn})
modulates to the carrier frequency $\omega_k$. However, the
transmitted signal may only use up a small portion of the total
available bandwidth. To increase bandwidth efficiency and
throughput, why not send additional QAM signals in other portions
of the unused bandwidth simultaneously. An example of this is
shown in Figure~\ref{oqamsys}, where we have taken the QAM
modulator of Figure~\ref{qam_mod} and put several of them in
parallel, each with a different carrier frequency. The data for
each modulator came from portions of a high speed bit stream. This
is principle behind multicarrier modulation.

\begin{description}
    \item[{\bf Question \arabic{questioncnt}\stepcounter{questioncnt}:}]
    Draw a schematic for an orthogonally multiplexed QAM
    demodulator. Show that the subcarriers of the orthogonally multiplexed
    QAM system are orthogonal.
\end{description}


\begin{description}
    \item[{\bf Matlab Task \arabic{matlabcnt}\stepcounter{matlabcnt}:}]
    Implement Figure~\ref{oqamsys} and its corresponding receiver.
    Verify that it works under ideal conditions. What carrier
    frequencies did you not use in the implementation?
\end{description}


Orthogonal Frequency Division Multiplexing (OFDM) is an efficient
type of multicarrier modulation, which employs the discrete
Fourier transform (DFT) and inverse DFT (IDFT) to modulate and
demodulate the data streams. The set-up of an OFDM system is
presented in Figure~\ref{ofdmsys}. A high-speed digital input,
$x[m]$, is demultiplexed into $N$ subcarriers using a commutator.
The data on each subcarrier is then modulated into an $M$-QAM
symbol, which maps a group of $\log_2(M)$ bits at a time. Unlike
the baseband representation of Eq.~(\ref{basic_qam_defn}), for
subcarrier $k$ we will rearrange $a^k$ and $b^k$ into real and
imaginary components such that the output of the modulator is
$p^k=a^k+jb^k$. In order for the output of the IDFT block to be
real, given $N$ subcarriers we must use a $2N$-point IDFT, where
terminals $k=0$ and $k=N$ are ``don't care'' inputs. For the
subcarriers $1\le{k}\le{N-1}$, the inputs are $p^k=a^k+jb^k$,
while for the subcarriers $N+1\le{k}\le{2N-1}$, the inputs are
$p^{k}=a^{2N-k}+jb^{2N-k}$.

\begin{description}
    \item[{\bf Question \arabic{questioncnt}\stepcounter{questioncnt}:}]
    Why are $k=0$ and $k=N$ ``don't care'' inputs?
\end{description}

The IDFT is then performed, yielding
\begin{equation}
s[n]=\frac{1}{2N}\sum\limits_{k=0}^{2N-1}p^ke^{j(2\pi{nk}/2N)}
\end{equation}
which results in the data being transmitted on several
subchannels. Figure~\ref{freq_resp_ofdm} shows the subchannels on
which the data is being transmitted on. The modulation to these
subchannels is achieved since each data stream is multiplied by a
$\sin(Nx)/\sin(x)$, several of which are shown in
Figure~\ref{ofdm_tones}.

The subcarriers are then multiplexed together using a commutator,
forming the signal $y[n]$, and transmitted to the receiver. Once
at the receiver, the signal is demultiplexed into $2N$ subcarriers
of data, $\hat{s}[n]$, using a commutator and a $2N$-point DFT,
defined as
\begin{equation}
\bar{p}^k=\sum\limits_{n=0}^{2N-1}\hat{s}[n]e^{-j(2\pi{nk}/2N)},
\end{equation}
is applied to the inputs, yielding the estimates of $p^k$,
$\bar{p}^k$. The output, $\hat{p}^k$, then passed through a
demodulator and the result multiplexed together using a
commutator, yielding the reconstructed high-speed bit stream,
$\hat{x}[m]$.
\begin{figure}[h]
\centering
\includegraphics{images/ofdm_system.eps}\\
\caption{Overall schematic of an Orthogonal Frequency Division
Multiplexing System}\label{ofdmsys}
\end{figure}


\begin{figure}[h]
\begin{center}
\subfigure[Frequency response of OFDM
subcarriers]{\label{freq_resp_ofdm}\includegraphics{images/f8.eps}}\goodgap
\subfigure[Real parts of the OFDM basis functions for subcarriers
1, 2, 4, and
7]{\label{ofdm_tones}\includegraphics[width=2.5in,keepaspectratio=true]{images/f9.eps}}
\caption{Characteristics of Orthogonal Frequency Division
Multiplexing}
\end{center}
\end{figure}

\begin{description}
    \item[{\bf Question \arabic{questioncnt}\stepcounter{questioncnt}:}]
    Show that the OFDM configuration and the orthogonally
    multiplexed QAM system are identical. Which of the two
    implementations is faster if the OFDM system employs FFTs?
    Prove it by comparing the number of operations.
\end{description}

\begin{description}
    \item[{\bf Matlab Task \arabic{matlabcnt}\stepcounter{matlabcnt}:}]
    Implement Figure~\ref{ofdmsys} using IFFT and FFT blocks.
\end{description}



\section{Channel Environment}

Until now, we have only considered an OFDM system operating under
ideal conditions (i.e. the transmitter is connected directly to
the receiver without any introduced distortion). We have already
dealt with the concept of noise in Section~\ref{noisesubsection}.
Now we will cover channel distortion due to dispersive propagation
and how it is compensated for in OFDM system.


\subsection{Dispersive Propagation}

From our high school physics courses, we learned about wave
propagation and how they combine constructively and destructively.
The exact same principles hold in high speed data transmission.

For example, in a wireless communication system, the transmitter
emanates radiation in all directions (unless the antenna is
``directional'', in which case, the energy is focused at a
particular azimuth). In an open environment, like a barren farm
field, the energy would continue to propagate until some of it
reaches the receiver antenna. As for the rest of the energy, it
continues on until it dissipates.

In an indoor environment, as depicted in
Figure~\ref{room_response}, the situation is different. The
line-of-sight component (if it exists), $p_1$, arrives at the
receiver antenna first, just like in the open field case. However,
the rest of the energy does not simply dissipate. Rather, the
energy is reflected by the walls and other objects in the room.
Some of these reflections, such as $p_2$ and $p_3$, will make
their way to the receiver antenna, although not with the same
phase or amplitude. All these received components are functions of
several parameters, including their overall distance between the
transmitter and receiver antennas as well as the number of
reflections. At the receiver, these components are just copies of
the same transmitted signal, but with different arrival times,
amplitudes, and phases. Therefore, one can view the channel as an
impulse response that is being convolved with the transmitted
signal. In the open field case, the {\it channel impulse response}
(CIR) would be a delta, since no other copies would be received by
the receiver antenna. On the other hand, an indoor environment
would have a several copies intercepted at the receiver antenna,
an thus its CIR would be similar to the example in
Figure~\ref{cir_diagram}. The corresponding frequency response of
the example CIR is shown in Figure~\ref{cir_freq_resp}.

In an xDSL environment, the same principles can be applied to the
wireline environment. The transmitted signal is sent across a
network of telephone wires, with numerous junctions, bridging
taps, and connections to other customer appliances (e.g.
telephones, xDSL modems). If the impedances are not matched well
in the network, reflections occur and will reach the devices
connected to the network, including the desired receiver.

With the introduction of the CIR, new problems arise in our
implementation which need to be addressed. In
Section~\ref{ofdmcp}, we will look at how to undo the smearing
effect the CIR has on the transmitted signal. In
Section~\ref{sceq}, we will look at one technique employed
extensively in OFDM which inverts the CIR effects in the frequency
domain.

\begin{figure}[h]
\begin{center}
\subfigure[Impulse
response]{\label{cir_diagram}\includegraphics{images/f10.eps}}\goodgap
\subfigure[Frequency
response]{\label{cir_freq_resp}\includegraphics{images/f11.eps}}\\
\subfigure[The process by which dispersive propagation
arises]{\label{room_response}\includegraphics{images/room.eps}}
\caption{Example of a channel response due to dispersive
propagation}
\end{center}
\end{figure}

\begin{description}
    \item[{\bf Matlab Task \arabic{matlabcnt}\stepcounter{matlabcnt}:}]
    Implement the channel filter, given as an input the channel
    impulse response, and add the noise generator.
\end{description}



\subsection{OFDM with Cyclic Prefix}\label{ofdmcp}

The CIR can be modelled as a finite impulse response filter that
is convolved with a sampled version of the transmitted signal. As
a result, the CIR smears past samples onto current samples, which
are smeared onto future samples. The effect of this smearing
causes distortion of the transmitted signal, increasing the
aggregate BER of the system and resulting in a loss in
performance.

Although equalizers can be designed to undo the effects of the
channel, there is a trade-off between complexity and distortion
minimization that is associated with the choice of an equalizer.
In particular, the distortion due to the smearing of a previous
OFDM symbol onto a successive symbol is a difficult problem. One
simple solution is to put a few ``dummy'' samples between the
symbols in order to capture the intersymbol smearing effect. The
most popular choice for these $K$ dummy samples are the last $K$
samples of the current OFDM symbol. The dummy samples in this case
is known as a {\it cyclic prefix}, as shown in
Figure~\ref{cyclicprefix_1}.

Therefore, when the OFDM symbols with cyclic prefixes are passed
through the channel, the smearing from the previous symbols are
captured by the cyclic prefixes, as shown in
Figure~\ref{cyclicprefix_2}. As a result, the symbols only
experience smearing of samples from within their own symbol.

At the receiver, the cyclic prefix is removed, as shown in
Figure~\ref{cyclicprefix_3}, and the OFDM symbols proceed with
demodulation and equalization.

Despite the usefulness of the cyclic prefix, there are several
disadvantages. First, the length of the cyclic prefix must be
sufficient to capture the effects of the CIR. If not, the cyclic
prefix fail to prevent distortion introduced from other symbols.
The second disadvantage is the amount of overhead introduced by
the cyclic prefix. By adding more samples to buffer the symbols,
we must send more information across the channel to the receiver.
This means to get the same throughput as a system without the
cyclic prefix, we must transmit at a higher data rate.

\begin{figure}[h]
\begin{center}
\subfigure[Add cyclic prefix to an OFDM symbol]{\label{cyclicprefix_1}\includegraphics[width=4.5in,keepaspectratio=true]{images/cyclicprefix_1.eps}}\\
\subfigure[Smearing by channel h(t) from previous symbol into cyclic prefix]{\label{cyclicprefix_2}\includegraphics[width=4.5in,keepaspectratio=true]{images/cyclicprefix_2.eps}}\\
\subfigure[Removal of cyclic
prefix]{\label{cyclicprefix_3}\includegraphics[width=4.5in,keepaspectratio=true]{images/cyclicprefix_3.eps}}
\caption{The process of adding, smearing capturing, and removal of
a cyclic prefix}
\end{center}
\end{figure}

\begin{description}
    \item[{\bf Matlab Task \arabic{matlabcnt}\stepcounter{matlabcnt}:}]
    Implement the cyclic prefix add and remove blocks of the OFDM
    system.
\end{description}


\section{Frequency Domain Equalization}\label{sceq}

Once the cyclic prefix has taken care of the intersymbol
interference, the received signal has the DFT applied to it.
However, the smearing between samples still exist and must be
compensated for. This is achieved by multiplying the subcarriers
with the inverse of the channel frequency response.

Referring to Figures~\ref{ofdm_tones} and \ref{cir_freq_resp}, if
we multiply the two figures together, we notice that each of the
subcarriers have a different gain. Therefore, what must be done is
to multiply each subcarrier with a gain that is an inverse to the
channel frequency response acting on that subcarrier. This is the
principle behind {\it per tone equalization}. Knowing what the
channel frequency gains are at the different subcarriers, one can
use them to reverse the distortion caused by the channel by
dividing the subcarriers with them. For instance, if the system
has 64 subcarriers centered at frequencies $\omega_k=2\pi{k}/64$,
$k=0,\ldots,63$, then one would take the CIR $h[n]$ and take its
64-point FFT, resulting with the frequency response $H[k]$,
$k=0,\ldots,63$. Then, to reverse the effect of the channel on
each subcarrier, simply take the inverse of the channel frequency
response point corresponding to that subcarrier,
\begin{equation}
W[k]=\frac{1}{H[k]}
\end{equation}
and multiply the subcarrier with it.

\begin{description}
    \item[{\bf Matlab Task \arabic{matlabcnt}\stepcounter{matlabcnt}:}]
    Implement the per tone equalization block. How was the complex gain chosen?
\end{description}



\section{Bit Allocation}

Most OFDM systems use the same signal constellation across all
subcarriers, as shown in Figure~\ref{ccommutator}, where the
commutator allocates bit groupings of the same size to each
subcarrier. However, their overall error probability is dominated
by the subcarriers with the worst performance. To improve
performance, adaptive bit allocation can be employed, where the
signal constellation size distribution across the subcarriers
varies according to the measured signal-to-noise ratio (SNR)
values, as shown in Figure~\ref{vcommutator}, where the commutator
allocates bit groupings of different sizes. In extreme situations,
some subcarriers can be ``turned off'' or {\it nulled} if the
subcarrier SNR values are poor.

Given that both the transmitter and receiver possess knowledge of
the subcarrier SNR levels, one is able to determine the subcarrier
BER values (see Figure~\ref{ber_curves}). Since the subcarrier BER
values are dependent on the choice of modulation used for a given
SNR value, we can vary the modulation scheme used in each
subcarrier in order to change the subcarrier BER value. Larger
signal constellations (e.g. 64-QAM) require larger SNR values in
order to reach the same BER values as smaller constellations (e.g.
4-QAM) which have smaller SNR values.

We will simply allocate bits according to
\begin{equation}\label{shannon}
b_i=\log_2\Big({1+\frac{\gamma_i}{\Gamma}}\Big),
\end{equation}
where $\gamma_i$ is the SNR of subcarrier $i$. Assuming equal
energy across all used subcarriers, the SNR Gap $\Gamma$ is
adjusted until the target bit rate is exceeded.



\begin{figure}[h]
\begin{center}
\subfigure[Constant-Rate
Commutator]{\label{ccommutator}\includegraphics{images/commutator_conventional.eps}}\goodgap
\subfigure[Variable-Rate
Commutator]{\label{vcommutator}\includegraphics{images/commutator_adaptive.eps}}
\caption{Comparison of variable and constant rate commutators with
equivalent total rate}
\end{center}
\end{figure}


\begin{description}
    \item[{\bf Question \arabic{questioncnt}\stepcounter{questioncnt}:}]
    Find the expression for the subcarrier SNR, given the channel
    impulse response, the subcarrier transmit power levels, and
    the noise variance $\sigma^2$.
\end{description}

\begin{description}
    \item[{\bf Matlab Task \arabic{matlabcnt}\stepcounter{matlabcnt}:}]
    Implement the bit allocation scheme as per Eq.~(\ref{shannon}).
\end{description}



\end{document}
